\section{TFT Display}
The Communication Interface with the TFT display is a 5V TTL UART with 250.000 baud of standard 8N1 protocol.

\subsection{HBC to TFT}
For sending commands or instructions to the TFT display, everything needs to be in ASCII followed by 3 Bytes of 0xFF. 

Nearly all commands directly set properties of the elements placed on screen. For example setting the value of a floating point number (here: Element name x1) with two decimal places to the value ``3.30'' would be as follows:

``\textcolor{orange}{x1.val=330}''

Note, that no decimal point is set. The display can only handle integer numbers. Decimal points are handled by the corresponding number box element on screen. 

Here is an overview of the most important commands for Interfacing. All examples are made with textbox t1 and float number x1.

\begin{table}[H]
    \centering
    \begin{tabular}{|c|l|}
        \hline
        \textbf{Command}  & \multicolumn{1}{|c|}{\textbf{Description}}\\ \hline \hline
        t1.txt=``foo'' & set text \\ \hline
        t1.aph=127 & set opacity to fully visible \\ \hline
        x1.val=330 & set numeric value \\ \hline
        vis t1,0 & set textbox invisible. a 1 at the end makes it visible again \\ \hline
    \end{tabular}
\label{tab:tft_overview}
\end{table}

\subsection{Change Device Page}
So far, every module class has its own page on the display. So only these modules need to be actively monitored. 

If the user changes the page, the following command will be sentby the TFT: 

``\textcolor{orange}{0x55 0x\#\#}''where \#\# is the following number:

\begin{table}[H]
    \centering
    \begin{tabular}{|c|l|}
        \hline
        \textbf{Number}  & \multicolumn{1}{|c|}{\textbf{Description}}\\ \hline \hline
        00h & Home Screen / General Settings\\ \hline
        01h & Diode Tester \\ \hline
        02h & Function Generator \\ \hline
        03h & Symmetrical Power Supply \\ \hline
        04h & Switching Mode Power Supply \\ \hline
    \end{tabular}
\label{tab:tft_pagecmd}
\end{table}

\subsection{Set Input Process Data}
Depending on the module, this command can have a variable length of data trailing it. The general instruction across all modules begins with ``\textcolor{orange}{0x5C}''.

\begin{table}[H]
    \centering
    \begin{tabular}{|c|l|}
        \hline
        \textbf{length}  & \multicolumn{1}{|c|}{\textbf{Description}}\\ \hline \hline
        1 byte & Module ID\\ \hline
        1 byte & Operation mode \\ \hline
        1 byte & Start Current Index \\ \hline
        1 Byte & Stop Current Index \\ \hline
    \end{tabular}
\caption[Data Structure Diode Tester]{Data Structure Diode Tester}
\label{tab:tft_diode_5C}

\end{table}
\begin{table}[H]
    \centering
    \begin{tabular}{|c|l|}
        \hline
        \textbf{length}  & \multicolumn{1}{|c|}{\textbf{Description}}\\ \hline \hline
        1 byte & Module ID\\ \hline
        4 byte & Frequency - LSB first \\ \hline
        2 byte & Amplitude - LSB first \\ \hline
        2 byte & Offset - LSB first \\ \hline
    \end{tabular}
\caption[Data Structure Function Generator]{Data Structure Function Generator}
\label{tab:tft_fgen_5C}
\end{table}

\begin{table}[H]
    \centering
    \begin{tabular}{|c|l|}
        \hline
        \textbf{length}  & \multicolumn{1}{|c|}{\textbf{Description}}\\ \hline \hline
        1 byte & Module ID\\ \hline
        2 byte & Positive Voltage - LSB first \\ \hline
        2 byte & Positive Current - LSB first \\ \hline
        2 byte & Negative Voltage - LSB first \\ \hline
        2 byte & Negative Current - LSB first \\ \hline
    \end{tabular}
\caption[Data Structure Symmetric Power Supply]{Data Structure Symmetric Power Supply}
\label{tab:tft_sympsu_5C}
\end{table}

\subsection{Set Output}
The Set Output instruction only exists for modules that actually have a disconnectable output. The command generally follows the format ``\textcolor{orange}{0x50 0xIS}'', where I is the module sub-ID (numbered 1-4) and S is the state of the output (0 = off, 1 = on). 

\subsection{General Settings (Home Screen)}
General settings are mostly for debug purposes, but can be activated and deactivated on the home screen. The instruction begins with ``\textcolor{orange}{0x5A}'' and is followed by the data below:

\begin{table}[H]
    \centering
    \begin{tabular}{|c|l|}
        \hline
        \textbf{Data}  & \multicolumn{1}{|c|}{\textbf{Description}}\\ \hline \hline
        10h & Disable CAN-Log over USB \\ \hline
        11h & Enable CAN-Log over USB \\ \hline
        20h & Disable TFT-Log over USB \\ \hline
        21h & Enable TFT-Log over USB \\ \hline
        30h & Don't simulate missing modules \\ \hline
        31h & Simulate all missing modules \\ \hline
    \end{tabular}
\label{tab:tft_5A}
\end{table}